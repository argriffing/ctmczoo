% to build, do something like
% rubber --pdf four-state


\documentclass{article}

\usepackage{graphics}
\usepackage{tikz}
\usepackage{amsmath}
\usepackage{mathtools}

% required for \blacksquare
\usepackage{amssymb}

\usetikzlibrary{matrix,positioning}

% label the states
\providecommand{\Ao}{\text{a}}
\providecommand{\Bo}{\text{b}}
\providecommand{\Co}{\text{c}}
\providecommand{\Do}{\text{d}}

% label the two parameters
\providecommand{\Pa}{\alpha}
\providecommand{\Pb}{\beta}

% for drawing block matrix structure
\providecommand{\bs}{\blacksquare}
\providecommand{\bx}{\Box}
\providecommand{\cd}{\cdot}

% helper functions for parenthesized sums
\providecommand{\psab}{\left( \alpha + \beta \right)}
\providecommand{\psaab}{\left( 2 \alpha + \beta \right)}
\providecommand{\psabb}{\left( \alpha + 2 \beta \right)}


% the default max matrix dimensions in amsmath is ten
\setcounter{MaxMatrixCols}{20}

% this is copied from the internet
% and its purpose is to compensate for bad vertical spacing in amsmath pmatrix
% by letting you do things like \begin{pmatrix}[1.5]
\makeatletter
\renewcommand*\env@matrix[1][\arraystretch]{%
	  \edef\arraystretch{#1}%
	    \hskip -\arraycolsep
	      \let\@ifnextchar\new@ifnextchar
	        \array{*\c@MaxMatrixCols c}}
		\makeatother


\begin{document}


%\title{Expected rate vs. ``spectral rate''}
%\author{Alex Griffing}
%\maketitle

\section{Introduction}

In this technical note we review properties of a particular four-state
continuous-time Markov chain over states $\{ \Ao, \Bo, \Co, \Do \}$
with positive rates, deliberately over-parameterized with
three variables so that the effects of scaling either or both of the rates
can be easily observed.
%
FIXME
\begin{equation}
\bordermatrix{
	    & \Ao  & \Bo & \Co & \Do \cr
	\Ao & \bs  & t \Pa & \cdot & foo \cr
	\Bo & t \Pb & \bs & t \Pb & foo \cr
	\Co & \cdot & t \Pa & \bs & foo  \cr
	\Do & \cdot & t \Pa & foo & \bs  \cr
}
\end{equation}

\section{Stationary distribution}

Because the rate matrix is symmetric,
the stationary distribution is uniform over the four states.

\section{Time reversibility}

It is easy to check that this rate matrix is time-reversible.

\section{Expected rate}

FIXME

The total expected rate is
$\frac{4 t \Pa \Pb}{\Pa + 2 \Pb}$.
Note that the expected rate is proportional to the rate matrix scaling
factor $t$.
This proportionality is true in general, not just for this particular model.

\section{Spectrum}
% eigenvalues {{-t (a+b), t b, 0, t a}, {t b, -t(a+b), ta , 0},
% {0, t a, -t(a+b), t b}, {t a, 0, t b, -t(a+b)}}

The eigenvalues of the rate matrix are
$\{ 0, -2 t \Pa, -2 t \Pb, -2 t \left( \Pa + \Pb \right) \}$.
The nonzero eigenvalue
$\lambda_2 \left( \Pa, \Pb, t \right)$
is related to the eventual rate
at which information about the initial state is lost.
Note that this eigenvalue is the smaller in magnitude of
$-2 t \Pa$ and $-2 t \Pb$ and is proportional to the rate matrix scaling
factor $t$.
Again, this proportionality is true in general,
not just for this particular model.

\section{Transition matrix}

The transition matrix is the matrix exponential of the rate matrix.
% MatrixExp[ {{-t (a+b), t b, 0, t a}, {t b, -t(a+b), ta , 0},
% {0, t a, -t(a+b), t b}, {t a, 0, t b, -t(a+b)}} ]
\begin{equation}
	\frac{1}{4} e^{-4 t \psab}
	\begin{pmatrix}
		e^{2t \psab} + e^{4t \psab} + e^{2t \psaab} + e^{2t \psabb} &
		-e^{2t \psab} + e^{4t \psab} - e^{2t \psaab} + e^{2t \psabb} &
		e^{2t \psab} + e^{4t \psab} - e^{2t \psaab} - e^{2t \psabb} &
		-e^{2t \psab} + e^{4t \psab} + e^{2t \psaab} - e^{2t \psabb} \\
		%
		-e^{2t \psab} + e^{4t \psab} - e^{2t \psaab} + e^{2t \psabb} &
		e^{2t \psab} + e^{4t \psab} + e^{2t \psaab} + e^{2t \psabb} &
		-e^{2t \psab} + e^{4t \psab} + e^{2t \psaab} - e^{2t \psabb} &
		e^{2t \psab} + e^{4t \psab} - e^{2t \psaab} - e^{2t \psabb} \\
		%
		e^{2t \psab} + e^{4t \psab} - e^{2t \psaab} - e^{2t \psabb} &
		-e^{2t \psab} + e^{4t \psab} + e^{2t \psaab} - e^{2t \psabb} &
		e^{2t \psab} + e^{4t \psab} + e^{2t \psaab} + e^{2t \psabb} &
		-e^{2t \psab} + e^{4t \psab} - e^{2t \psaab} + e^{2t \psabb} \\
		%
		-e^{2t \psab} + e^{4t \psab} + e^{2t \psaab} - e^{2t \psabb} &
		e^{2t \psab} + e^{4t \psab} - e^{2t \psaab} - e^{2t \psabb} &
		-e^{2t \psab} + e^{4t \psab} - e^{2t \psaab} + e^{2t \psabb} &
		e^{2t \psab} + e^{4t \psab} + e^{2t \psaab} + e^{2t \psabb}
	\end{pmatrix}
\end{equation}

\end{document}
