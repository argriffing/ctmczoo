% to build, do something like
% rubber --pdf three-state


\documentclass{article}

\usepackage{graphics}
\usepackage{tikz}
\usepackage{amsmath}
\usepackage{mathtools}

% required for \blacksquare
\usepackage{amssymb}

\usetikzlibrary{matrix,positioning}

% label the states
\providecommand{\Ao}{\text{a}}
\providecommand{\Bo}{\text{b}}
\providecommand{\Co}{\text{c}}
\providecommand{\Do}{\text{d}}

% label the two parameters
\providecommand{\Pa}{\alpha}
\providecommand{\Pb}{\beta}

% for drawing block matrix structure
\providecommand{\bs}{\blacksquare}
\providecommand{\bx}{\Box}
\providecommand{\cd}{\cdot}

% the default max matrix dimensions in amsmath is ten
\setcounter{MaxMatrixCols}{20}

% this is copied from the internet
% and its purpose is to compensate for bad vertical spacing in amsmath pmatrix
% by letting you do things like \begin{pmatrix}[1.5]
\makeatletter
\renewcommand*\env@matrix[1][\arraystretch]{%
	  \edef\arraystretch{#1}%
	    \hskip -\arraycolsep
	      \let\@ifnextchar\new@ifnextchar
	        \array{*\c@MaxMatrixCols c}}
		\makeatother


\begin{document}


%\title{Expected rate vs. ``spectral rate''}
%\author{Alex Griffing}
%\maketitle

\section{Introduction}

In this technical note we review properties of a particular three-state
continuous-time Markov chain over states $\{ \Ao, \Bo, \Co \}$
with positive rates, deliberately over-parameterized with
three variables so that the effects of scaling either or both of the rates
can be easily observed.
%
\begin{equation}
\bordermatrix{
	    & \Ao  & \Bo & \Co \cr
	\Ao & \bs  & t \Pa & \cdot \cr
	\Bo & t \Pb & \bs & t \Pb \cr
	\Co & \cdot & t \Pa & \bs  \cr
}
\end{equation}

\section{Stationary distribution}

The stationary distribution
$\left( \pi_\Ao, \pi_\Bo, \pi_\Co \right)$
should satisfy
\begin{align}
0 &=
\left( \pi_\Ao, \pi_\Bo, \pi_\Co \right)
\begin{pmatrix}
	-t \Pa & t \Pa & 0 \\
	t \Pb & - 2 t \Pb & t \Pb \\
	0 & t \Pa & -t \Pa
\end{pmatrix} \\
0 &< \pi_{i} \\
1 &= \pi_\Ao + \pi_\Bo + \pi_\Co
\end{align}
%
so
\begin{align}
\left( \pi_\Ao, \pi_\Bo, \pi_\Co \right)
&=
\left(
	\frac{\Pb}{\Pa + 2 \Pb},
	\frac{\Pa}{\Pa + 2 \Pb},
	\frac{\Pb}{\Pa + 2 \Pb}
\right) \\
&\propto
\left( \Pb, \Pa, \Pb \right)
\end{align}
%
Because the model is in continuous time
and each state can reach every other state,
we do not need to worry about subtleties involving reducibility or periodicity.

\section{Time reversibility}

It is easy to check that this rate matrix is time-reversible,
with expected rates out of (and into) states $\Ao$ and $\Co$
of $\frac{t \Pa \Pb}{\Pa + 2 \Pb}$ each,
and with expected rate out of (and into) state $\Bo$
of $\frac{2 t \Pa \Pb}{\Pa + 2 \Pb}$.

\section{Expected rate}

The total expected rate is
$\frac{4 t \Pa \Pb}{\Pa + 2 \Pb}$.
Note that the expected rate is proportional to the rate matrix scaling
factor $t$.
This proportionality is true in general, not just for this particular model.

\section{Spectrum}

The eigenvalues of the rate matrix are
% eigenvalues {{-t a, t a, 0}, {t b, - (t b + t b), t b}, {0, t a, - t a}}
$\{ 0, - t \Pa, - t \left( \Pa + 2 \Pb \right) \}$
The nonzero eigenvalue
$\lambda_2 \left( \Pa, \Pb, t \right)$
is related to the eventual rate
at which information about the initial state is lost.
Note that this eigenvalue is proportional to the rate matrix scaling
factor $t$.
Again, this proportionality is true in general,
not just for this particular model.

\section{Transition matrix}

The transition matrix is the matrix exponential of the rate matrix.
% MatrixExp[{{-t a, t a, 0}, {t b, - 2 t b, t b}, {0, t a, - t a}}]
\begin{equation}
	\frac{1}{\Pa + 2 \Pb}
	\begin{pmatrix}
		\frac{1}{2} e^{-t \Pa} \left(
			\Pa e^{-2 t \Pb} + 2 \Pb e^{t \Pa} + \Pa + 2 \Pb
		\right) &
		\Pa - \Pa e^{-t \left( \Pa + 2 \Pb \right)} &
		\frac{1}{2} e^{-t \Pa} \left(
			\Pa e^{-2 t \Pb} + 2 \Pb e^{t \Pa} - \Pa - 2 \Pb
		\right) \\
		\Pb - \Pb e^{ - t \left( \Pa + 2 \Pb \right) } &
		\Pa + 2 \Pb e^{ - t \left( \Pa + 2 \Pb \right) } &
		\Pb - \Pb e^{ - t \left( \Pa + 2 \Pb \right) } \\
		\frac{1}{2} e^{-t \Pa} \left(
			\Pa e^{-2 t \Pb} + 2 \Pb e^{t \Pa} - \Pa - 2 \Pb
		\right) &
		\Pa - \Pa e^{-t \left( \Pa + 2 \Pb \right)} &
		\frac{1}{2} e^{-t \Pa} \left(
			\Pa e^{-2 t \Pb} + 2 \Pb e^{t \Pa} + \Pa + 2 \Pb
		\right)
	\end{pmatrix}
\end{equation}

\end{document}
