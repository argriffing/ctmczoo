% to build, do something like
% rubber --pdf two-state

\documentclass{article}

\usepackage{graphics}
\usepackage{tikz}
\usepackage{amsmath}
\usepackage{mathtools}

% required for \blacksquare
\usepackage{amssymb}

\usetikzlibrary{matrix,positioning}

% label the states
\providecommand{\Ao}{\text{a}}
\providecommand{\Bo}{\text{b}}
\providecommand{\Co}{\text{c}}
\providecommand{\Do}{\text{d}}

% label the two parameters
\providecommand{\Pa}{\alpha}
\providecommand{\Pb}{\beta}

% for drawing block matrix structure
\providecommand{\bs}{\blacksquare}
\providecommand{\bx}{\Box}
\providecommand{\cd}{\cdot}

% the default max matrix dimensions in amsmath is ten
\setcounter{MaxMatrixCols}{20}

% this is copied from the internet
% and its purpose is to compensate for bad vertical spacing in amsmath pmatrix
% by letting you do things like \begin{pmatrix}[1.5]
\makeatletter
\renewcommand*\env@matrix[1][\arraystretch]{%
	  \edef\arraystretch{#1}%
	    \hskip -\arraycolsep
	      \let\@ifnextchar\new@ifnextchar
	        \array{*\c@MaxMatrixCols c}}
		\makeatother


\begin{document}


%\title{Expected rate vs. ``spectral rate''}
%\author{Alex Griffing}
%\maketitle

\section{Introduction}

In this technical note we review properties of the two-state continuous-time
Markov chain over states $\{ \Ao, \Bo \}$ with positive rates
$\Ao \to \Bo = t \Pa$ and $\Bo \to \Ao = t \Pb$.
This is deliberately over-parameterized with three variables so that
the effects of scaling either or both of the rates can be easily observed.
%
\begin{equation}
\bordermatrix{
	    & \Ao  & \Bo  \cr
	\Ao & \bs  & t \Pa \cr
	\Bo & t \Pb & \bs  \cr
}
\end{equation}
%
In more complicated models the rate matrix representation is simplified
by black squares on the diagonal representing values that force
rows to sum to zero, but for the simple 2-state model this notation
does not buy us much so we can write the rate matrix more explicitly.
%
\begin{equation}
\bordermatrix{
	    & \Ao  & \Bo  \cr
	\Ao & - t \Pa  & t \Pa \cr
	\Bo & t \Pb & - t \Pb  \cr
}
\end{equation}

\section{Stationary distribution}

The stationary distribution
$\left( \pi_\Ao, \pi_\Bo \right)$
should satisfy
\begin{align}
0 &=
\left( \pi_\Ao, \pi_\Bo \right)
\begin{pmatrix}
	- t \Pa &  t \Pa \\
	t \Pb & - t \Pb
\end{pmatrix} \\
0 &< \pi_\Ao \\
0 &< \pi_\Bo \\
1 &= \pi_\Ao + \pi_\Bo
\end{align}
%
so
\begin{align}
\left( \pi_\Ao, \pi_\Bo \right)
&=
\left( \frac{\Pb}{\Pa + \Pb}, \frac{\Pa}{\Pa + \Pb} \right) \\
&\propto
\left( \Pb, \Pa \right)
\end{align}
%
Because the model is in continuous time
and each state can reach every other state,
we do not need to worry about subtleties involving reducibility or periodicity.

\section{Time reversibility}

A model is said to be time-reversible when it satisfies the
`detailed balance' equations which define a kind of `pairwise equilibrium'.
Because this model has only two states, the `pairwise equilibrium'
is the same as the overall equilibrium, so this model is time-reversible.

\section{Expected rate}

This scalar function of the rate matrix is the expected rate
away from the current state.
For this model, it is
\begin{align}
\text{expected rate}
&=
\left( \pi_\Ao t \Pa + \pi_\Bo t \Pb \right) \\
&=
\frac{2 t \Pa \Pb}{\Pa + \Pb}
\end{align}
Note that the expected rate is proportional to the rate matrix scaling
factor $t$.
This proportionality is true in general, not just for this particular model.

\section{Spectrum}

The eigenvalues of the rate matrix are
$\{ 0, - t \left( \Pa + \Pb \right) \}$.
The nonzero eigenvalue
$\lambda_2 \left( \Pa, \Pb, t \right)$
is related to the eventual rate
at which information about the initial state is lost.
Note that this eigenvalue is proportional to the rate matrix scaling
factor $t$.
Again, this proportionality is true in general,
not just for this particular model.

\section{Transition matrix}

The transition matrix is the matrix exponential of the rate matrix.
\begin{equation}
	\frac{e^{- t \left( \Pa + \Pb \right)}}{\Pa + \Pb}
	\begin{pmatrix}
		\Pb e^{t \left( \Pa + \Pb \right) } + \Pa &
		\Pa e^{t \left( \Pa + \Pb \right) } - \Pa \\
		\Pb e^{t \left( \Pa + \Pb \right) } - \Pb &
		\Pa e^{t \left( \Pa + \Pb \right) } + \Pb
	\end{pmatrix}
\end{equation}

\end{document}
